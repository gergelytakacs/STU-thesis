\documentclass{book}


\begin{document}

Zlomky piseme pomocou funkcie
\begin{eqnarray}
G(s)=\frac{1}{s^2}
\end{eqnarray}

Toto je normalna zatvorka pri matike $a(b+2)$ ale takisto toto je v poriadku  $a[b+2]$.

Ak mame komplikovanejsie vyrazy, musime pouzivat specialne znaky na zatvorky ako
\begin{eqnarray}
G(s) = K\left( \frac{1}{s^2}  \right)
\end{eqnarray}
alebo
\begin{eqnarray}
G(s) = K\left[ \frac{1}{s^2}  \right]
\end{eqnarray}

Indexy funguju aj tu samozrejme:
\begin{eqnarray}
G(s) = \left( \frac{1}{s^2}  \right)^N_{i=1}
\end{eqnarray}


\end{document} 