\documentclass{book}

\begin{document} 

Tabulky definujeme s prostredim „tabular“. Vedla tabulky mozeme zadefinovat aj zarovnanie buniek:

\begin{tabular}{ l c r }
  1 & 2 & 3 \\
  4 & 5 & 6 \\
  7 & 8 & 9 \\
\end{tabular}

Ciarky mozeme pridat nasledovne

\begin{tabular}{| l  | c | r | }
  1 & 2 & 3 \\
  \hline
  4 & 5 & 6 \\
  7 & 8 & 9 \\
\end{tabular}

Ak chceme definovat umiestnenie v dokumente s tym ze davame aj popis aj referenciu tak musime
\begin{table}
\centering
\caption{Popis tabulky}
\begin{tabular}{ l   c  r  }
  1 & 2 & 3 \\
  \hline
  4 & 5 & 6 \\
  7 & 8 & 9 \\
\end{tabular}
\label{mojatabulka}
\end{table}

Umiestnenie tabulky mozeme aj nutit:
\begin{table}[!h]
\centering
\caption{Popis tabulky}
\begin{tabular}{ l   c  r  }
  1 & 2 & 3 \\
  \hline
  4 & 5 & 6 \\
  7 & 8 & 9 \\
\end{tabular}
\label{mojatabulka}
\end{table}




\end{document} 