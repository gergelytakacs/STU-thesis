\documentclass{book}
\usepackage{amsmath}

\begin{document}


Premenna $A$ je sucastou textu, tak ako aj rovnica $1+1=2$.

Potom mozeme mat aj samostatne a cislovane rovnice, ktore piseme do prostredia ``eqnarray'':
\begin{eqnarray}
1+1=2
\end{eqnarray}

Ak chceme mat viacero rovnic, tak
\begin{eqnarray}
1+1&=&2\\
4-1&=&3\\
1+1&=&2
\end{eqnarray}

Ak nechceme cislovat rovnicu tak
\begin{eqnarray}
1+1&=&2\\
\nonumber
4-1&=&3\\
1+1&=&2
\end{eqnarray}

Moja rovnica je Rov. (\ref{mojarovnica}) alebo ked mame balik AMS Math potom moja rovnica je Rov. \eqref{mojarovnica}.
\begin{eqnarray}
\label{mojarovnica}
1+1=2
\end{eqnarray}



\end{document} 